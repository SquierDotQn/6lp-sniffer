\chapter{Déroulement du projet}
\label{sec:deroulement}

\section{Prise en main du sujet et des technologies}
	Après ces explications techniques, il est temps d'expliquer comment le projet s'est déroulé pour nous. Au début, il nous a fallu prendre en main les différents outils pour travailler.
	
	\subsection{Contiki}
	Comme nous avons déjà vu, Contiki est un système d'exploitation qui vise les systèmes embarqués. Notre expérience avec l'embarqué était limitée bien que non-nulle, aussi la prise en main s'est faite assez rapidement, même si difficilement.\\
	Certains concepts, comme les proto-threads utilisés par Contiki, sont assez proches d'autres concepts présents dans les systèmes d'exploitation habituels, néanmoins la découverte des fonctionnalités de Contiki a pris du temps car il est complet et offre les principales caractéristiques et fonctionnalités d'un système habituel.\\
	Ceci est l'une des raisons, avec celles déjà évoquées pour IPv6 et 6LoWPAN, du choix de Contiki plutôt que FreeRTOS ou TinyOS.
%		\begin{itemize}
%			\item Embarqué
%			\item simulations avec Cooja
%			\item Pourquoi Contiki et pas un autre ?
%		\end{itemize}

	\subsection{Chaîne de compilation et pilotes}
	Les découvertes ici ont été plutôt rapides puisque la machine virtuelle InstantContiki3.0 contient la chaîne de compilation nécessaire à la compilation des projets sur les différentes architectures supportées par Contiki. L'installation à la main de la chaîne de compilation est bien documentée sur les différents tutoriels concernant Contiki présents sur Internet.\\
	Les pilotes des différents contrôleurs radios sont par contre difficiles à prendre en main, aussi avons nous choisi de nous concentrer sur les contrôleurs cc2420 de Texas Instrument car déjà présent dans d'autres projets que nous avons passés en revue.
%		\begin{itemize}
%			\item GCC
%			\item architectures différentes
%			\item contrôleurs radio différents
%		\end{itemize}

\section{Programme développé}
	\subsection{Fonctionnement du projet}
	\subsection{État du projet}

\section{Retours d'expérience} % DONE ?
    Pendant ce projet, nous avons pu apprendre beaucoup de choses, développer nos compétences et de produire un programme, dont qui peut bénéficier d'évolutions sur le court et long terme.
    
	\subsection{Évolutions à court terme} % DONE
	A court terme, il est évidemment possible d'ajouter d'autres détections d'attaques en fonction des besoins, mais ce qui est le plus utile à l'équipe de recherche est d'intégrer notre sonde dans Discus, leur système de détection d'intrusion.
	La sonde peut fournir les informations dont Discus a besoin pour faire respecter les contraintes énoncée dans le script adéquat. 
	Les vérifications d'attaques sont donc effectée par le système qui reçoit les informations, et non par les sondes directement, ce qui allège la charge de travail sur les capteurs aux capacités restreintes. Aussi, les nouvelles vérifications d'intrusion pourront s'écrire et se faire sans reprogrammer les sondes.
	\subsection{Évolutions à long terme} % DONE
	Sur un plus long terme, il a été pensé d'éventuellement faire communiquer les sondes entre elles afin de créer une grille de capteurs.
	Cela permettrait, grâce aux différents RSSI pour un seul paquet captés par les sondes, de localiser les différents acteurs du réseau, et donc de localiser l'attaquant lors d'une anomalie.
	\subsection{Challenges} % DONE
	    Ce projet a été très intéressant et instructif, mais nous avons dû faire face à plusieurs difficultés et challenges lors de son déroulement. Bien que ces contraintes aient pu parfois nous ralentir, elles furent instructives à plusieurs niveaux.\\
	    Tout d'abord, le sujet du projet se concentre sur des domaines et technologies dont nous n'étions pas très familiers. Le monde de l'informatique embarquée est fait de contraintes auxquelles il faut s'adapter pour être productif, les retours beaucoup moins verbeux lors d'erreurs, les limites de mémoire, de puissance, et parfois l'absence de librairies pour rendre le code assez léger pour la plateforme sont quelques exemples de difficultés lorsqu'on découvre l'embarqué. S'y adapter n'est pas un obstacle en soi, mais il faut bien se préparer et ne pas avoir peur de prendre son temps pour cela.\\
	    Les découvertes étaient nettement plus nombreuses dans les technologies employées, notamment au niveau des systèmes d'exploitation embarqués, comme Contiki, et leurs technologies de communication. Nous avons lu beaucoup de spécifications, de RFC et de documentation, et en rétrospective, nous aurions eu plus de facilité à établir un plan d'approche, organiser nos découvertes pour éviter la confusion. Par exemple, prendre du temps pour bien se renseigner sur Contiki, puis lorsque l'outil est maîtrisé, se renseigner sur 6LoWPAN, et ainsi de suite.\\
	    Malgré nos recherches en profondeur, nous sommes parfois tombés sur des incohérences dans la documentation, ou des explications pas assez claires, notamment sur le buffer de paquets qui traite différemment les paquets réseau selon qu'ils soient entrants ou sortants. Pour pallier à ce souci, nous avons recherché plus d'informations sur des sites et des encyclopédies en ligne (wikis) universitaires qui se sont penchés sur Contiki et ont écrit de bons tutoriels.\\
	    Parfois, certains exemples de code présents dans Contiki ne sont pas assez commentés, ce qui peut être problématique pour apprendre comment certains programmes sont faits. C'est pourquoi il ne faut pas hésiter à rechercher sur des forums des discussions traitant de ces sujets et à contacter certains interlocuteurs pour demander de l'aide.

%%% Local Variables: 
%%% mode: latex
%%% TeX-master: "isae-report-template"
%%% End: 
