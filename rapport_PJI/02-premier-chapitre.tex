\chapter{Contexte du sujet}
\label{chap:contexte}
%%\begin{itemize}
	%%\item Expliquer le projet
Dans notre projet nous devions réaliser des noeuds qui peuvent écouter le trafic réseau. Ce terme noeud n'apparait que dans le mode simulation du projet, en réalité on parle de sondes. Ce projet s'inscrit donc dans le domaine de l'Internet des objets. La description des outils utilisés et des stratégies mises en place pour surveiller le réseau sera détaillée dans les parties ci dessous. 
%%\end{itemize}

\section{Analyse de l'existant}
	
	\subsection{Internet des objets}
		%%\begin{itemize}
			%%\item Qu'est-ce que c'est ?
			L'Internet des Objets (ou Web of Things en anglais) correspond à l'extension d'internet aux éléments ou lieux du monde physique. Cette technologie est largement implantée dans notre société avec diverses applications comme la domotique, les achats par téléphone mobile, gestion des dechets.. Notre projet s'inscrit donc dans cet univers puisque les sondes intéragissent avec différents éléments réseaux d'un batiment par exemple. L'Internet des objets regroupe également différents modes de communications tels que le wifi, bluetooth.??definition de 6LowPAN\\
			%%\item Technologies utilisables (bluetooth, wifi, 6lowpan)
			Détailler les différentes technologies
		%%\end{itemize}
		
	\subsection{Technologies de communication}
		%%\begin{itemize}
			%\item 6lowpan qu'est-ce que c'est ?
			\begin{itemize}
				\item définition
				\item exemples (Linky, domotique, industrie)
			\end{itemize}
		\end{itemize}
		
	\subsection{Sécurité des communications}
		\begin{itemize}
			\item Sécurité dans 6lowpan
			\begin{itemize}
				\item RFC définissent sécurité
				\item Dans les faits, pas vraiment mis en place
			\end{itemize}
		\end{itemize}



\section{Objectif du projet}

	\subsection{Détail du projet}
		%%\begin{itemize}
		Nous avons expliqué le fonctionnement du projet mais non pourquoi il fallait mettre en place ces sondes et comment cela fonctionnait. Les noeuds ou "mote" vont analyser le trafic circulant, les données qu'elles peuvent récupérer peuvent faciliter la détéction d'intrusion. En effet les paquets émis contiennent des identifiants, des tailles spécifiées, des numéros de ports sur lesquels ils sont envoyés. Ces informations doivent être donc traitées avec rigueur et méthode car le système de communication est restreint (de part sa mémoire limitée) mais également au niveau de sa puissance.
			%%\item mote qui sniffe, oui mais quoi ?
			%%\item système contraint en puissance et en mémoire
		%%\end{itemize}
	\subsection{Où s'inscrit le projet ?}
		%%\begin{itemize}
			Comme énoncé précédemment, l'internet des objets peut s'appliquer dans de nombreux domaines. Les sondes 6LoWPAN s'adaptent donc à ces différents secteurs pour la surveillance en industrie par exemple, l'optimisation de calculs ou réaliser des mesures (thermiques, lumineuses ..). 
			%%\item Industrie, hopitaux, mobilier urbain ( pas "simple domotique" )
		%%\end{itemize}

\section{Réponse à un besoin de l'équipe}
	
	\subsection{Focus sur la sécurité par 2XS}
	\begin{itemize}
		\item Systèmes sûrs
		\begin{itemize}
			\item Preuves formelles
			\item Différents projets de sécurité
		\end{itemize}
		\item D'accord, mais quid des communications ?
	\end{itemize}
	
	\subsection{Discus}
	\begin{itemize}
		\item IDS -- Système de détection d'intrusion
	\end{itemize}

\section{Technologies et systèmes utilisés}
	Ici on présente les technologies utilisées
	\subsection{Contiki OS}
	\subsection{Outils de simulations}
		\begin{itemize}
			\item Cooja
			\item InstantContiki3.0
		\end{itemize}
	\subsection{Langage C embarqué et sa chaine de compilation spécifique}
	\subsection{Git}
	Git est un gestionnaire de version de projet. Celui ci permet de synchroniser le travail de notre binôme ainsi que d'autres documents. Nous avons stocké notre projet sur GitHub qui facilite la lecture de nos contributions.
%%% Local Variables: 
%%% mode: latex
%%% TeX-master: "isae-report-template"
%%% End: 
