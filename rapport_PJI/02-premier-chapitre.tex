\chapter{Contexte du sujet}
\label{chap:contexte}

\begin{itemize}
	\item Expliquer le projet
\end{itemize}

\section{Analyse de l'existant}
	
	\subsection{Internet des objets}
		\begin{itemize}
			\item Qu'est-ce que c'est ?	
			\item Technologies utilisables ( bluetooth, wifi, 6lowpan )
		\end{itemize}
		
	\subsection{Technologies de communication}
		\begin{itemize}
			\item 6lowpan qu'est-ce que c'est ?
			\begin{itemize}
				\item définition
				\item exemples (Linky, domotique, industrie)
			\end{itemize}
		\end{itemize}
		
	\subsection{Sécurité des communications}
		\begin{itemize}
			\item Sécurité dans 6lowpan
			\begin{itemize}
				\item RFC définissent sécurité
				\item Dans les faits, pas vraiment mis en place
			\end{itemize}
		\end{itemize}



\section{But du projet}

	\subsection{Détail du projet}
		\begin{itemize}
			\item mote qui sniffe, oui mais quoi ?
			\item système contraint en puissance et en mémoire
		\end{itemize}
	\subsection{Où s'inscrit le projet ?}
		\begin{itemize}
			\item Industrie, hopitaux, mobilier urbain ( pas "simple domotique" )
		\end{itemize}

\section{Réponse à un besoin de l'équipe}
	
	\subsection{Focus sur la sécurité par 2XS}
	
	\subsection{Disqus}

\section{Technologies et systèmes utilisés}
	Ici on présente les technologies utilisées
	\subsection{Contiki OS}
	\subsection{Outils de simulations}
		\begin{itemize}
			\item Cooja
			\item InstantContiki3.0
		\end{itemize}
	\subsection{Langage C embarqué et sa chaine de compilation spécifique}
	\subsection{Git}
%%% Local Variables: 
%%% mode: latex
%%% TeX-master: "isae-report-template"
%%% End: 