\chapter*{Conclusion et perspectives}
\addcontentsline{toc}{chapter}{Conclusion}
\markboth{Conclusion}{Conclusion}
\label{sec:conclusion}

    %Conclusion et ouverture
	%\item Résumer le sujet, sa problématique et notre morceau de solution
    Notre sujet de projet était de contribuer à un système de détection d'intrusion pour l'Internet des Objets.
    La principale difficulté de ce sujet était les nombreuses attaques possibles sur le réseau, notamment le détournement de routage pour l'écoute, le vol d'identité et la falsification de paquets. En partant de cette problématique, nous avons produit une sonde, un noeud sur le réseau qui écoute le trafic radio, afin de construire par dessus plusieurs couches de détections.\\
	%\item Ce qu'on a apporté
    De ce fait, nous procurons à l'équipe de recherche les prémisses d'un outil de base pour développer la sécurité réseau dans l'IDS de l'équipe -- Discus, en rajoutant la plateforme radio aux autres déjà existantes.\\

	%\item Ce que ça nous a apporté
	En travaillant sur ce projet, nous avons pu découvrir des technologies que nous n'avons pas l'habitude de voir lors de nos cours. Cela nous a permis d'élargir nos horizons vis à vis des domaines de l'informatique, de tester de nouveaux paradigmes.\\
	%\item Comment ce qu'on a appris va nous servir plus tard
	Nous nous sommes améliorés sur notre travail en autonomie ainsi que sur recherche d'information, de documentation. Cet entraînement est aussi un regard vers ce qu'est le monde de la recherche, qui a permis de le découvrir ou le redécouvrir.
	
	% TODO

%%% Local Variables: 
%%% mode: latex
%%% TeX-master: "isae-report-template"
%%% End: 

