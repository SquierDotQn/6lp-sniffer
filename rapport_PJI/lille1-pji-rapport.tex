% A LaTeX (non-official) template for ISAE projects reports
% Copyright (C) 2014 Damien Roque
% Version: 0.2
% Author: Damien Roque <damien.roque_AT_isae.fr>

\documentclass[a4paper,12pt]{book}
\usepackage[utf8]{inputenc}
\usepackage[T1]{fontenc}
\usepackage[frenchb]{babel} % If you write in French
%\usepackage[english]{babel} % If you write in English
\usepackage{a4wide}
\usepackage{graphicx}
\graphicspath{{images/}}
\usepackage{subfig}
\usepackage{tikz}
\usetikzlibrary{shapes,arrows}
\usepackage{pgfplots}
\pgfplotsset{compat=newest}
\pgfplotsset{plot coordinates/math parser=false}
\newlength\figureheight
\newlength\figurewidth
\pgfkeys{/pgf/number format/.cd,
set decimal separator={,\!},
1000 sep={\,},
}
\usepackage{ifthen}
\usepackage{ifpdf}
\ifpdf
\usepackage[pdftex]{hyperref}
\else
\usepackage{hyperref}
\fi
\usepackage{color}
\hypersetup{%
colorlinks=true,
linkcolor=black,
citecolor=black,
urlcolor=black}

\renewcommand{\baselinestretch}{1.05}
\usepackage{fancyhdr}
\pagestyle{fancy}
\fancyfoot{}
\fancyhead[LE,RO]{\bfseries\thepage}
\fancyhead[RE]{\bfseries\nouppercase{\leftmark}}
\fancyhead[LO]{\bfseries\nouppercase{\rightmark}}
\setlength{\headheight}{15pt}

\let\headruleORIG\headrule
\renewcommand{\headrule}{\color{black} \headruleORIG}
\renewcommand{\headrulewidth}{1.0pt}
\usepackage{colortbl}
\arrayrulecolor{black}

\fancypagestyle{plain}{
  \fancyhead{}
  \fancyfoot[C]{\thepage}
  \renewcommand{\headrulewidth}{0pt}
}

\makeatletter
\def\@textbottom{\vskip \z@ \@plus 1pt}
\let\@texttop\relax
\makeatother

\makeatletter
\def\cleardoublepage{\clearpage\if@twoside \ifodd\c@page\else%
  \hbox{}%
  \thispagestyle{empty}%
  \newpage%
  \if@twocolumn\hbox{}\newpage\fi\fi\fi}
\makeatother

\usepackage{amsthm}
\usepackage{amssymb,amsmath,bbm}
\usepackage{array}
\usepackage{bm}
\usepackage{multirow}
\usepackage[footnote]{acronym}

\newcommand*{\SET}[1]  {\ensuremath{\mathbf{#1}}}
\newcommand*{\VEC}[1]  {\ensuremath{\boldsymbol{#1}}}
\newcommand*{\FAM}[1]  {\ensuremath{\boldsymbol{#1}}}
\newcommand*{\MAT}[1]  {\ensuremath{\boldsymbol{#1}}}
\newcommand*{\OP}[1]  {\ensuremath{\mathrm{#1}}}
\newcommand*{\NORM}[1]  {\ensuremath{\left\|#1\right\|}}
\newcommand*{\DPR}[2]  {\ensuremath{\left \langle #1,#2 \right \rangle}}
\newcommand*{\calbf}[1]  {\ensuremath{\boldsymbol{\mathcal{#1}}}}
\newcommand*{\shift}[1]  {\ensuremath{\boldsymbol{#1}}}

\newcommand{\eqdef}{\stackrel{\mathrm{def}}{=}}
\newcommand{\argmax}{\operatornamewithlimits{argmax}}
\newcommand{\argmin}{\operatornamewithlimits{argmin}}
\newcommand{\ud}{\, \mathrm{d}}
\newcommand{\vect}{\text{Vect}}
\newcommand{\sinc}{\ensuremath{\mathrm{sinc}}}
\newcommand{\esp}{\ensuremath{\mathbb{E}}}
\newcommand{\hilbert}{\ensuremath{\mathcal{H}}}
\newcommand{\fourier}{\ensuremath{\mathcal{F}}}
\newcommand{\sgn}{\text{sgn}}
\newcommand{\intTT}{\int_{-T}^{T}}
\newcommand{\intT}{\int_{-\frac{T}{2}}^{\frac{T}{2}}}
\newcommand{\intinf}{\int_{-\infty}^{+\infty}}
\newcommand{\Sh}{\ensuremath{\boldsymbol{S}}}
%\newcommand{\C}{\SET{C}}
\newcommand{\R}{\SET{R}}
\newcommand{\Z}{\SET{Z}}
\newcommand{\N}{\SET{N}}
\newcommand{\K}{\SET{K}}
\newcommand{\reel}{\mathcal{R}}
\newcommand{\imag}{\mathcal{I}}
\newcommand{\cmnr}{c_{m,n}^\reel}
\newcommand{\cmni}{c_{m,n}^\imag}
\newcommand{\cnr}{c_{n}^\reel}
\newcommand{\cni}{c_{n}^\imag}
\newcommand{\tproto}{g}
\newcommand{\rproto}{\check{g}}
\newcommand{\LR}{\mathcal{L}_2(\SET{R})}
\newcommand{\LZ}{\ell_2(\SET{Z})}
\newcommand{\LZI}[1]{\ell_2(\SET{#1})}
\newcommand{\LZZ}{\ell_2(\SET{Z}^2)}
\newcommand{\diag}{\operatorname{diag}}
\newcommand{\noise}{z}
\newcommand{\Noise}{Z}
\newcommand{\filtnoise}{\zeta}
\newcommand{\tp}{g}
\newcommand{\rp}{\check{g}}
\newcommand{\TP}{G}
\newcommand{\RP}{\check{G}}
\newcommand{\dmin}{d_{\mathrm{min}}}
\newcommand{\Dmin}{D_{\mathrm{min}}}
\newcommand{\Image}{\ensuremath{\text{Im}}}
\newcommand{\Span}{\ensuremath{\text{Span}}}

\newtheoremstyle{break}
  {11pt}{11pt}%
  {\itshape}{}%
  {\bfseries}{}%
  {\newline}{}%
\theoremstyle{break}

%\theoremstyle{definition}
\newtheorem{definition}{Définition}[chapter]

%\theoremstyle{definition}
\newtheorem{theoreme}{Théorème}[chapter]

%\theoremstyle{remark}
\newtheorem{remarque}{Remarque}[chapter]

%\theoremstyle{plain}
\newtheorem{propriete}{Propriété}[chapter]
\newtheorem{exemple}{Exemple}[chapter]

\parskip=5pt
%\sloppy

\begin{document}

%%%%%%%%%%%%%%%%%%
%%% First page %%%
%%%%%%%%%%%%%%%%%%

\begin{titlepage}
\begin{center}

\includegraphics[width=0.6\textwidth]{logo_lille1}\\[1cm]

{\large Master 1 Informatique}\\[0.5cm]

{\large PJI - Projet Individuel - Sujet no 104}\\[0.5cm]

% Title
\rule{\linewidth}{0.5mm} \\[0.4cm]
{ \huge \bfseries Systèmes de détection d'intrusion pour l'Internet des Objets \\[0.4cm] }
\rule{\linewidth}{0.5mm} \\[1.5cm]

% Author and supervisor
\noindent
\begin{minipage}{0.4\textwidth}
  \begin{flushleft} \large
    \emph{Auteurs :}\\
    M. Théo \textsc{Plockyn}\\
    M. Rémy \textsc{Debue}
  \end{flushleft}
\end{minipage}%
\begin{minipage}{0.4\textwidth}
  \begin{flushright} \large
    \emph{Encadrant :} \\
    Pr.~Gilles \textsc{Grimaud}
  \end{flushright}
\end{minipage}

\vfill

% Bottom of the page
{\large Version 0.5 du\\ \today}

\end{center}
\end{titlepage}

%%%%%%%%%%%%%%%%%%%%%%%%%%%%%
%%% Non-significant pages %%%
%%%%%%%%%%%%%%%%%%%%%%%%%%%%%

\frontmatter

\chapter*{Remerciements}
Nous remercions tout d'abord l'équipe pédagogique, administrative et intervenants du Master 1 informatique de nous avoir encadré, aidé et assuré les enseignements dont nous avons disposé cette année.
\\ \\
Nous tenons aussi à remercier et à témoigner notre reconnaissance aux personnes suivantes :
\\ \\
Gilles Grimaud, notre encadrant, pour nous avoir proposé le sujet, nous avoir suivi et conseillé tout au long de ce projet.
\\ \\
Michaël Hauspie, pour ses consignes et sa participation dans les décisions du déroulement du projet.
\\ \\
Nadir Cherifi, pour son aide précieuse et ses connaissances des technologies utilisées qui nous ont débloqué à plusieurs reprises.
\\ \\
Samuel Hym, François Serman, Christophe Bacara, Quentin Bergougnoux, et toute l'équipe 2XS pour leur accueil sympathique et leur soutien tout au long de ce projet.

\clearpage
\tableofcontents

\clearpage
\listoffigures

\clearpage
\chapter*{Liste des sigles et acronymes}
\begin{acronym}[CP-OFDMX] % Give the longest acronym here
\acro{6LoWPAN}{\emph{IPv\textbf{6} \textbf{Lo}w power \textbf{W}ireless \textbf{P}ersonal \textbf{A}rea \textbf{N}etworks}}
\acro{LoWPAN}{\emph{\textbf{Lo}w power \textbf{W}ireless \textbf{P}ersonal \textbf{A}rea \textbf{N}etworks}}
\acro{IRCICA}{\textbf{I}nstitut de \textbf{r}echerche sur les \textbf{c}omposants logiciels et matériels pour l'\textbf{i}nformation et la \textbf{c}ommunication \textbf{a}vancée de Lille}
\acro{2XS}{\emph{e\textbf{X}tra \textbf{S}mall e\textbf{X}tra \textbf{S}afe} -- L'équipe de recherche}
\acro{CFS}{\emph{\textbf{C}offee \textbf{F}ile \textbf{S}ystem } -- Le système de fichier de Contiki}
\acro{DSL}{\emph{\textbf{D}omain \textbf{S}pecific \textbf{L}anguage} -- Langage dédié}
\acro{IDS}{\emph{\textbf{\textbf{I}ntrusion \textbf{D}etection \textbf{S}ystem}} -- Système de Detection d'Intrusions}
\acro{PJI}{\textbf{P}ro\textbf{j}et \textbf{i}ndividuel}
\acro{RFC}{\emph{\textbf{R}equest \textbf{F}or \textbf{C}omments} -- Documents de spécifications}
\acro{OS}{\emph{\textbf{O}perating \textbf{S}ystem} -- Système d'exploitation}
\end{acronym}

%%%%%%%%%%%%%%%%%%%%%%%%%%%%%%%%%%%%%%%%%%%%
%%% Content of the report and references %%%
%%%%%%%%%%%%%%%%%%%%%%%%%%%%%%%%%%%%%%%%%%%%

\mainmatter
\pagestyle{fancy}

\cleardoublepage

\chapter*{Introduction}
\addcontentsline{toc}{chapter}{Introduction}
\markboth{Introduction}{Introduction}
\label{chap:introduction}
%\minitoc

Dans le cadre de notre cursus en Master Informatique à Lille 1, nous avons eu l’opportunité de réaliser un projet sur l’ensemble du semestre appelé PJI. Chaque étudiant ou binôme pouvait choisir un sujet sur lequel travailler parmi une liste mais également proposer le sien.
Nous avons choisi de nous intéresser à un sujet proche de l'informatique embarquée, qui est un domaine grandissant à l'aube de l'Internet des Objets. Notre sujet se porte sur la détection de paquets falsifiés dans un réseau 6LoWPAN.

\begin{figure}[htp]
	\centering
	\includegraphics[width=16cm]{images/6lowpan.jpg}
	\caption{Diagramme d'explication de 6LoWPAN.}
	\label{fig:diagramme-6lowpan}
\end{figure}
L'équipe proposant ce sujet est le groupe 2XS \textbf{eXtra Small eXtra Safe} composée de notamment \textbf{Gilles GRIMAUD} notre encadrant, \textbf{Michael HAUSPIE} son collègue proche de ce sujet et bien sûr le reste de l'équipe.

%%% Local Variables: 
%%% mode: latex
%%% TeX-master: "isae-report-template"
%%% End: 

\chapter{Contexte du sujet}
\label{chap:contexte}
%%\begin{itemize}
	%%\item Expliquer le projet
Dans notre projet nous devions réaliser des noeuds qui peuvent écouter le trafic réseau. Ce terme noeud n'apparait que dans le mode simulation du projet, en réalité on parle de sondes. Ce projet s'inscrit donc dans le domaine de l'Internet des objets. La description des outils utilisés et des stratégies mises en place pour surveiller le réseau sera détaillée dans les parties ci dessous. 
%%\end{itemize}

\section{Analyse de l'existant}
	
	\subsection{Internet des objets}
		%%\begin{itemize}
			%%\item Qu'est-ce que c'est ?
			L'Internet des Objets (ou Web of Things en anglais) correspond à l'extension d'internet aux éléments ou lieux du monde physique. Cette technologie est largement implantée dans notre société avec diverses applications comme la domotique, les achats par téléphone mobile, gestion des dechets.. Notre projet s'inscrit donc dans cet univers puisque les sondes intéragissent avec différents éléments réseaux d'un batiment par exemple. L'Internet des objets regroupe également différents modes de communications tels que le wifi, bluetooth.??definition de 6LowPAN\\
			%%\item Technologies utilisables (bluetooth, wifi, 6lowpan)
			Détailler les différentes technologies
		%%\end{itemize}
		
	\subsection{Technologies de communication}
		%%\begin{itemize}
			%\item 6lowpan qu'est-ce que c'est ?
			\begin{itemize}
				\item définition
				\item exemples (Linky, domotique, industrie)
			\end{itemize}
		\end{itemize}
		
	\subsection{Sécurité des communications}
		\begin{itemize}
			\item Sécurité dans 6lowpan
			\begin{itemize}
				\item RFC définissent sécurité
				\item Dans les faits, pas vraiment mis en place
			\end{itemize}
		\end{itemize}



\section{Objectif du projet}

	\subsection{Détail du projet}
		%%\begin{itemize}
		Nous avons expliqué le fonctionnement du projet mais non pourquoi il fallait mettre en place ces sondes et comment cela fonctionnait. Les noeuds ou "mote" vont analyser le trafic circulant, les données qu'elles peuvent récupérer peuvent faciliter la détéction d'intrusion. En effet les paquets émis contiennent des identifiants, des tailles spécifiées, des numéros de ports sur lesquels ils sont envoyés. Ces informations doivent être donc traitées avec rigueur et méthode car le système de communication est restreint (de part sa mémoire limitée) mais également au niveau de sa puissance.
			%%\item mote qui sniffe, oui mais quoi ?
			%%\item système contraint en puissance et en mémoire
		%%\end{itemize}
	\subsection{Où s'inscrit le projet ?}
		%%\begin{itemize}
			Comme énoncé précédemment, l'internet des objets peut s'appliquer dans de nombreux domaines. Les sondes 6LoWPAN s'adaptent donc à ces différents secteurs pour la surveillance en industrie par exemple, l'optimisation de calculs ou réaliser des mesures (thermiques, lumineuses ..). 
			%%\item Industrie, hopitaux, mobilier urbain ( pas "simple domotique" )
		%%\end{itemize}

\section{Réponse à un besoin de l'équipe}
	
	\subsection{Focus sur la sécurité par 2XS}
	\begin{itemize}
		\item Systèmes sûrs
		\begin{itemize}
			\item Preuves formelles
			\item Différents projets de sécurité
		\end{itemize}
		\item D'accord, mais quid des communications ?
	\end{itemize}
	
	\subsection{Discus}
	\begin{itemize}
		\item IDS -- Système de détection d'intrusion
	\end{itemize}

\section{Technologies et systèmes utilisés}
	Ici on présente les technologies utilisées
	\subsection{Contiki OS}
	\subsection{Outils de simulations}
		\begin{itemize}
			\item Cooja
			\item InstantContiki3.0
		\end{itemize}
	\subsection{Langage C embarqué et sa chaine de compilation spécifique}
	\subsection{Git}
	Git est un gestionnaire de version de projet. Celui ci permet de synchroniser le travail de notre binôme ainsi que d'autres documents. Nous avons stocké notre projet sur GitHub qui facilite la lecture de nos contributions.
%%% Local Variables: 
%%% mode: latex
%%% TeX-master: "isae-report-template"
%%% End: 

\chapter{Explications techniques}
\label{sec:technique}

%% CONTIKI =================================================================
\section{Contiki}

	\subsection{Pile réseau de Contiki}
	
	\subsection{Systèmes de stockage Contiki}
	
		\subsubsection{Flash}
			\begin{itemize}
				\item Expliquer comment écrire dedans
				\item Expliquer pourquoi on ne l'a pas utilisé
			\end{itemize}
		\subsubsection{Volatile}
			\textbf{Listes}
			\begin{itemize}
				\item Expliquer comment écrire dedans
				\item Expliquer pourquoi on ne l'a pas utilisé
			\end{itemize}
			\textbf{Buffers cycliques}
			\begin{itemize}
				\item Expliquer comment écrire dedans
				\item Expliquer pourquoi on l'a utilisé
			\end{itemize}
		
	
%% 6LoWPAN =================================================================
\section{6LoWPAN}
	\subsection{Compression des headers}
		\begin{itemize}
			\item Pourquoi compresser ?
			\item Comment ça marche ( avec images )
		\end{itemize}
	
	
	\subsection{Attaques possibles}
	Nos possibles solutions du coup


%%% Local Variables: 
%%% mode: latex
%%% TeX-master: "isae-report-template"
%%% End: 

\chapter{Déroulement du projet}
\label{sec:deroulement}

\section{Prise en main du sujet et des technologies}
	Après avoir développé la partie sur les explications techniques, il serait judicieux de détailler le déroulement du projet pour notre binôme. Pour commencer, nous devions prendre en main les différents outils nécessaires au projet.
	
	\subsection{Contiki}
	Comme nous avons déjà vu, Contiki est un système d'exploitation qui est orienté systèmes embarqués. Notre expérience avec ce domaine était limitée puisque ce domaine n'est pas enseigné à l'université, du moins dans notre filière. Bien que l'interaction avec Contiki s'est faite assez rapidement, sa compréhension restait cependant difficile.\\
	Certains concepts, comme les proto-threads (parallélisation très légère avec une taille mémoire limitée) utilisés par Contiki, sont assez proches d'autres concepts présents dans les systèmes d'exploitation habituels, néanmoins la découverte des autres fonctionnalités de Contiki s'est faite au fur et à mesure. En effet Contiki possède de nombreuses caractéristiques en plus de celles qu'offrent les systèmes habituels.\\
	Ces notions, en plus de celles déjà évoquées sur IPv6 et 6LoWPAN, nous confortent dans le choix de Contiki plutôt que FreeRTOS ou TinyOS, qui regroupe l'ensemble des caractéristiques des autres OS.
%		\begin{itemize}
%			\item Embarqué
%			\item simulations avec Cooja
%			\item Pourquoi Contiki et pas un autre ?
%		\end{itemize}

	\subsection{Chaîne de compilation et pilotes}
	L'utilisation des chaines de compilations s'est faite rapidement puisque la machine virtuelle InstantContiki3.0 en possède déjà une pour les projets existants et leur différente architecture supportée par Contiki. Les différents tutoriels présents sur le web ont facilités l'installation manuelle de ces chaînes de compilation.\\
	En revanche, les pilotes des différents contrôleurs radios nous ont semblé difficiles à prendre en main. Nous avons également choisis de nous concentrer sur les contrôleurs cc2420 (utilisés pour les communications sans fils) de Texas Instrument puisqu'ils sont déjà présents dans les autres projets.
%		\begin{itemize}
%			\item GCC
%			\item architectures différentes
%			\item contrôleurs radio différents
%		\end{itemize}

\section{Programme développé}
	\subsection{Fonctionnement du projet}
	\subsection{État du projet}

\section{Retours d'expérience} % DONE ?
    Durant notre projet, nous avons eu l'opportunité d'améliorer nos compétences dans les différents domaines du web des objets. Ce travail pourra bénéficier d'évolutions sur le court comme sur le long terme.
    
	\subsection{Évolutions à court terme} % DONE
	Sur le court terme, il serait possible d'ajouter d'autres types de détections d'attaques en fonction des besoins, mais ces ajouts ne sont pas pertinent dans le cadre de notre projet. En effet l'équipe de recherche 2XS souhaite intégrer la sonde dans Discus qui est déjà un système de détection d'intrusion, ce qui rendrait ces modifications redondantes.
	La sonde pourrait également fournir les informations dont Discus a besoin pour faire respecter les contraintes énoncée dans le script adéquat. 
	De ce fait, les vérifications d'attaques seraient donc effectuées par le système qui reçoit les informations, et non plus par les sondes directement. Ceci permet d'alléger la charge de travail sur les capteurs aux capacités restreintes. Aussi, les nouvelles vérifications d'intrusion pourront êtres implémentées sans reprogrammer les sondes.
	\subsection{Évolutions à long terme} % DONE
	Sur un plus long terme, il a été pensé d'éventuellement faire communiquer les sondes entre elles afin de créer une grille de capteurs.
	Cela permettrait, grâce aux différents RSSI associés à un seul paquet capté, de localiser les différents acteurs du réseau, et donc de localiser l'attaquant lors d'une anomalie.
	\subsection{Challenges} % DONE
	    Ce projet a été enrichissant et concernait les domaines qui nous intéressent (comme l'embarqué et/ou la sécurité), mais nous avons dû faire face à plusieurs difficultés et challenges lors de son déroulement. Bien que ces contraintes ralentissaient l'avancement du projet, elles furent instructives à plusieurs niveaux.\\
	    Tout d'abord, le sujet se concentre sur des domaines et technologies dont nous n'étions pas très familiers. Le monde de l'informatique embarquée est fait de contraintes auxquelles il faut s'adapter pour être productif, les retours beaucoup moins verbeux lors d'erreurs, les limites de mémoire, de puissance, et parfois l'absence de librairies pour rendre le code assez léger pour la plateforme sont quelques exemples de difficultés lorsque l'on découvre l'embarqué. Pouvoir s'adapter peut paraître laborieux mais la solution est de bien organiser le contexte et prendre le temps nécessaire pour assimiler les bases.\\
	    Les découvertes étaient nettement plus nombreuses dans les technologies employées, notamment au niveau des systèmes d'exploitation embarqués, comme Contiki, et leurs technologies de communication. Nous avons lu beaucoup de spécifications, de RFC et de documentation et en rétrospective, nous aurions eu plus de facilité à établir un plan d'approche, organiser nos découvertes pour éviter la confusion. Par exemple, prendre du temps pour bien se renseigner sur Contiki, puis lorsque l'outil est maîtrisé, se renseigner sur 6LoWPAN, et continuer à procéder par étapes durant l'ensemble du pji.\\
	    Malgré nos recherches en profondeur, nous sommes parfois tombés sur des incohérences dans la documentation, ou des explications confuses, notamment sur le buffer de paquets qui n'a pas la même structure si les paquets sont entrants ou sortants. Pour pallier à ce souci, nous nous sommes documentés sur des sites et des encyclopédies en ligne (wikis) universitaires traitant de Contiki avec des tutoriels adaptés à notre besoin.\\
	    Une autre difficulté durant le projet venait de Contiki lui même. En effet, certains exemples de code déjà présents ne sont pas assez commentés, ceci peut être problématique pour assimiler le fonctionnement d'un programme. C'est pourquoi nous nous sommes rendus sur des forums traitant de ces sujets et avons sollicité des membres de l'équipe 2XS.

%%% Local Variables: 
%%% mode: latex
%%% TeX-master: "isae-report-template"
%%% End: 

\chapter*{Conclusion et perspectives}
\addcontentsline{toc}{chapter}{Conclusion}
\markboth{Conclusion}{Conclusion}
\label{sec:conclusion}

    Conclusion et ouverture
    \begin{itemize}
    	\item Résumer le sujet, sa problématique et notre morceau de solution
    	\item Ce qu'on a apporté
    	\item Ce que ça nous a apporté
    	\item Comment ce qu'on a appris va nous servir plus tard
    \end{itemize}

%%% Local Variables: 
%%% mode: latex
%%% TeX-master: "isae-report-template"
%%% End: 



\clearpage

\end{document}